\documentclass[conference]{IEEEtran}
% Some Computer Society conferences also require the compsoc mode option,
% but others use the standard conference format.
%
% If IEEEtran.cls has not been installed into the LaTeX system files,
% manually specify the path to it like:
% \documentclass[conference]{../sty/IEEEtran}





% Some very useful LaTeX packages include:
% (uncomment the ones you want to load)


% *** MISC UTILITY PACKAGES ***
%
%\usepackage{ifpdf}
% Heiko Oberdiek's ifpdf.sty is very useful if you need conditional
% compilation based on whether the output is pdf or dvi.
% usage:
% \ifpdf
%   % pdf code
% \else
%   % dvi code
% \fi
% The latest version of ifpdf.sty can be obtained from:
% http://www.ctan.org/pkg/ifpdf
% Also, note that IEEEtran.cls V1.7 and later provides a builtin
% \ifCLASSINFOpdf conditional that works the same way.
% When switching from latex to pdflatex and vice-versa, the compiler may
% have to be run twice to clear warning/error messages.






% *** CITATION PACKAGES ***
%
\usepackage{cite}




% *** GRAPHICS RELATED PACKAGES ***
%
\ifCLASSINFOpdf
  % \usepackage[pdftex]{graphicx}
  % declare the path(s) where your graphic files are
  % \graphicspath{{../pdf/}{../jpeg/}}
  % and their extensions so you won't have to specify these with
  % every instance of \includegraphics
  % \DeclareGraphicsExtensions{.pdf,.jpeg,.eps}
\else
  % or other class option (dvipsone, dvipdf, if not using dvips). graphicx
  % will default to the driver specified in the system graphics.cfg if no
  % driver is specified.
  % \usepackage[dvips]{graphicx}
  % declare the path(s) where your graphic files are
  % \graphicspath{{../eps/}}
  % and their extensions so you won't have to specify these with
  % every instance of \includegraphics
  % \DeclareGraphicsExtensions{.eps}
\fi

\usepackage{graphicx}
\graphicspath{{figs/}}

% correct bad hyphenation here
\hyphenation{op-tical net-works semi-conduc-tor}

\begin{document}
%
% paper title
% Titles are generally capitalized except for words such as a, an, and, as,
% at, but, by, for, in, nor, of, on, or, the, to and up, which are usually
% not capitalized unless they are the first or last word of the title.
% Linebreaks \\ can be used within to get better formatting as desired.
% Do not put math or special symbols in the title.
\title{Symmetric Travelling Salesman Problem Using Parallel Simulated Annealing}


% author names and affiliations
% use a multiple column layout for up to three different
% affiliations
\author{\IEEEauthorblockN{Runchao Han}
\IEEEauthorblockA{School of Computer Science\\
The University of Manchester\\
Manchester, UK\\
Email: runchao.han@student.manchester.ac.uk}
}


% make the title area
\maketitle

% As a general rule, do not put math, special symbols or citations
% in the abstract
\begin{abstract}



\end{abstract}




% For peer review papers, you can put extra information on the cover
% page as needed:
% \ifCLASSOPTIONpeerreview
% \begin{center} \bfseries EDICS Category: 3-BBND \end{center}
% \fi
%
% For peerreview papers, this IEEEtran command inserts a page break and
% creates the second title. It will be ignored for other modes.
\IEEEpeerreviewmaketitle



\section{Introduction}

Travelling Salesman Problem (TSP)\cite{lawler1985traveling} is a problem easy to describe but hard to solve. It can be simply described as: Given a list of cities and their coordinates, what is the best route for a travelling salesman to to through all cities by the shortest path? 

TSP represents a larger class of problems known as {\it combinatorial optimization problems}. The TSP problem belongs in the class of such problems known as NP-complete\cite{leeuwen1990handbook}. Therefore, optimisation algorithms are usually applied to solve TSP, like Ant Colony, Hill Climbing and Simulated Annealing\cite{lawler1985traveling}.


This report proposed %TODO

\section{Related Work}

\subsection{Solving TSP by Simulated Annealing}

Simulated Annealing is an optimisation algorithm to optimise an initial solution by changing a small part of a solution iteratively. In eact iteration the modification contributes to the solution. If the solution is better, the modification will be accepted. If the solution is worser, a possibility is given that determines if the solution will be accepted or not\cite{reinelt1994traveling}.

TSP can be solved by Simulated Annealing, where a pair of cities are swapped in each iteration, as described in \cite{reinelt1994traveling}.

\subsection{Parallelising Simulated Annealing}

\cite{ram1996parallel} designed two parallelised Simulated Annealing schemes. The first one is to start multiple tasks and make them compare their results periodically, where the best result is chosen for the next period for all tasks, called the Cluster Algorithm, while the second one combines Genetic Algorithm with Simulated Annealing. 

\section{Design and Implementation}

\subsection{The Sequential Simulated Annealing}

\subsection{The Parallel Simulated Annealing}

%TODO better implementation


\section{Benchmark}

\section{Conclusion}


\bibliographystyle{IEEEtran}
\bibliography{miniproject.bib}


% that's all folks
\end{document}


